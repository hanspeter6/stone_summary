\documentclass{article}
\usepackage[utf8]{inputenc}

\title{summary on stone}
\author{hanspeter6 }
\date{November 2019}

\usepackage{natbib}

\begin{document}

\maketitle

\section{Introduction}

1 Projection Pursuit

1.1 p71

\begin{itemize}
    \item  Signal mixtures tend to have gaussian pdf and source signals do not
    \item A source signal can be extracted (unmixed) by taking the inner product of a weight vector and the signal mixtures where it provides an orthogonal projection of the signal mixtures. $s=Wx$.
\end{itemize}

But, how do we find W?, the unmixing matrix (vector)

One way is exploratory projection pursuit (or just projection pursuit). This aims to find one projection at a time such that the extracted signal is as non-gaussian as possible. The name is based on the notion that the method aims to find a weight vector that provides an orthogonal projection of a set of signal mixtures such that each signal has a pdf that is as non-gaussian as possible.

Another is independent component analysis. This generally extracts M signals simultaneously from M signal mixtures. This could require the estimating of a large MxM matrix of weights. An obvious benefit of projectio pursuit is that its possible to extract fewer than M signals.

As an example: height of an individual is considered to be the sum of a genetic component and a dietary one. We assume the contributions of each is the same for all individuals (ie the ration nature/nurture is constant. That is: $h^{i}=as_{G}^{i}+bs_{D}^{i}$, representing the formation of a signal mixture h from a linear combination of the two source signals, using mixing coefficients a and b, which are the same for all individuals (ie the relative contributions of diet and genetics are the same). The CLT would suggest that the pdf of $h_{i}$ would be approximately gaussian irrespective of the pdfs of the individual contributions $s_{G}^{i}$ and $s_{D}^{i}$ .

1.2 p72

Its important to note that the converse of the CLT is not generally true. That is, it is not true that any gaussian mixture is a mixture of non-gaussian signals, although in practice this is true, ie., that a gaussian mixture consists of a mixture of non-gaussian signals. 

So, in summary: given a set of gaussian mixtures, we can find each source signal by finding the unmixing vector that extracts the most non-gaussian signal. One strategy for doing this is to define a measure of “non-gausianity” and then finding the unmixing vector that maximises this measure.

Two types: super-gaussian and sub-gaussian or platykurtotic and leptokurtotic, respectively. The first in contrast to the second has most of its values clustered around zero. (proj pursuit methods based on Kullback-Leibler divergence can extract source signals from mixtures of super- and sub-gaussian (see source: FastICA by Hyvarinen et al, 2001a) Compare with the R methods???

1.3 p.73

Kurtosis is a measure of “peakyness”. So, we would try to find an unmixing vector that maximises the kurtosis of an extracted signal $y=w^{T}$ . But, first a closer look at kurtosis.

Kurtosis is defined as:

$$K=\frac{\frac{1}{N}\sum_{t=1}^{N}(\bar{y}-y^{t})^{4}}{(\frac{1}{N}\sum_{t=1}^{N}(\bar{y}-y^{t})^{2})^{2}}-3$$

The numerator is the important term and the constant (3) ensures that gaussian signals have zero kurtosis. Super-gaussian signals have positive kurtosis and sub-gaussian signals have negative kurtosis. The denominator is simply the variance of y, ensuring the kurtosis takes account of the signal variance.

\section{Projection pursuit}

\cite{stone2004independent} contrasts projection pursuit methods with ICA as the former seeking one projection at a time such that the extracted signal is as non-gaussian as possible, while the latter typically extracting $M$ signals simultaneously from $M$ signal mixtures.

\bibliographystyle{apalike}
\bibliography{references}

\end{document}
